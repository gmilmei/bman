\documentclass[12pt]{report}

\usepackage[a4paper,margin=2.5cm]{geometry}
\usepackage{booktabs}
\usepackage{graphicx}

\begin{document}

\parindent=0cm
\parskip=1ex

\newcommand{\f}[1]{\ensuremath{\mathit{#1}}}

\begin{titlepage}
  \newgeometry{left=7.5cm}
  \noindent
  {\huge\textbf{\textsf{User's Reference to B}}}\\[1em]
  \makebox[0pt][l]{\rule{1.3\textwidth}{1pt}}
  \par
  \vfill
  \noindent
  {\LARGE\textsf{K. Thompson}}
  \vskip\baselineskip
  \noindent
  \textsf{January 7, 1972}
\end{titlepage}

\newcommand{\sect}[2]{\section*{\large{#1\hspace{0.5em}#2}}}
\newcommand{\token}[1]{\texttt{\underline{#1}}}

\begin{abstract}
  B is a computer language intended for recursive, primarily non-
  numeric applications typified by system programming. B has a small,
  unrestrictive syntax that is easy to compile. Because of the unusual
  freedom of expression and a rich set of operators, B programs are
  often quite compact.

  This manual contains a concise definition of the language, sample
  programs, and instructions for using the PDP-11 version of
  B.\footnote{The cover sheet and distribution list have been omitted. [ed]}
\end{abstract}

\newpage

\begin{center}
  \large\textbf{MEMORANDUM FOR FILE}
\end{center}

\sect{1.0}{Introduction}

B is a computer language directly descendant from BCPL [1,2].  B is
running at Murray Hill on the DEC PDP-11 computer under the UNIX-11
time sharing system [3,4]. B is good for recursive, non-numeric,
machine independent applications, such as system and language work.

B, compared to BCPL, is syntactically rich in expressions and
syntactically poor in statements. A look at the examples in section~9
of this document will give a flavor of the language.

B was designed and implemented by D. M. Ritchie and the author.

\sect{2.0}{Syntax}

The syntactic notation in this manual is basically BNF with the
following exceptions:
\begin{enumerate}
\item The metacharacters $<$ and $>$ are removed.  Literals are
  underlined to differentiate them from syntactic
  variables.\footnote{Additionally, literals are typeset in
    \texttt{monospace} font. [ed]}
\item The metacharacter $|$ is removed.  Each syntactic alternative is
  placed on a separate line.
\item The metacharacters $\{$ and $\}$ denote syntactic grouping.  A
  syntactic group followed by numerical sub- and superscripts denote
  repetition of the group as follows:

  \begin{tabular}{ll}
     $\{ \ldots \}_m$ & $m, m+1, \ldots$ \\
     $\{ \ldots \}_m^n$ & $m, m+1, \ldots, n$ \\
  \end{tabular}
\end{enumerate}

\sect{2.1}{Canonical Syntax}

The syntax given in this section defines all the legal constructions
in B without specifying the association rules. These are given later
along with the semantic description of each construction.
\begin{quote}
  \baselineskip=20pt
  \begin{tabbing}
    \hspace{1cm} \= \kill
  program $::=$ \\
  \> $\{$definition$\}_0^{}$ \\
  definition $::=$ \\
  \> name $\{$ \token{[} $\{$constant$\}_0^1$ \token{]} $\}_0^1$
  $\{$ival $\{$\token{,} ival$\}_0^{}$$\}_0^1$ \token{;} \\
  \> name \token{(} $\{$name $\{$\token{,} name$\}_0^{}\}_0^1$ \token{)}
  statement \\
  ival $::=$ \\
  \> constant \\
  \> name \\
  statement $::=$ \\
  \> \token{auto}
  name
  $\{$constant$\}_0^1$
  $\{$\token{,} name $\{$constant$\}_0^1\}_0^{}$
  \token{;} statement
  \\
  \> \token{extrn}
  name
  $\{$\token{,} name$\}_0^{}$  
  \token{;} statement
  \\
  \> name \token{:} statement \\
  \> \token{case} constant \token{:} statement \\
  \> \token{\{} $\{$statement$\}_0$  \token{\}} \\
  \> \token{if} \token{(} rvalue \token{)} statement
  $\{$\token{else} statement$\}_0^1$ \\
  \> \token{while} \token{(} rvalue \token{)} statement \\
  \> \token{switch} rvalue statement \\
  \> \token{goto} rvalue \token{;} \\
  \> \token{return} $\{$\token{(} rvalue \token{)}$\}_0^1$ \token{;} \\
  \> $\{$rvalue$\}_0^1$ \token{;} \\
  rvalue $::=$ \\
  \> \token{(} rvalue \token{)} \\
  \> lvalue \\
  \> constant \\
  \> lvalue assign rvalue \\
  \> inc-dec lvalue \\
  \> lvalue inc-dec \\
  \> unary rvalue \\
  \> \token{\&} lvalue \\ 
  \> rvalue binary rvalue \\
  \> rvalue \token{?} rvalue \token{:} rvalue \\
  \> rvalue \token{(} $\{$rvalue $\{$\token{,} rvalue$\}_0^{}\}_0^1$ \token{)} \\
  assign $::=$ \\
  \> \token{=} $\{$binary$\}_0^1$ \\
  inc-dec $::=$ \\
  \> \token{++} \\
  \> \token{--} \\
  unary $::=$ \\
  \> \token{-} \\
  \> \token{!} \\
  binary $::=$ \\
  \> \token{|} \\
  \> \token{\&} \\ % TODO
  \> \token{==} \\
  \> \token{!=} \\
  \> \token{<} \\
  \> \token{<=} \\
  \> \token{>} \\
  \> \token{>=} \\
  \> \token{<<} \\
  \> \token{>>} \\
  \> \token{+} \\
  \> \token{-} \\
  \> \token{\%} \\
  \> \token{*} \\
  \> \token{/} \\
  lvalue $::=$ \\
  \> name \\
  \> \token{*} rvalue \\
  \> rvalue \token{[} rvalue \token{]} \\
  constant $::=$ \\
  \> $\{$digit$\}_1^{}$ \\
  \> \token{'} $\{$char$\}_1^2$ \token{'} \\
  \> \token{"} $\{$char$\}_0^{}$ \token{"} \\
  name $::=$ \\
  \> alpha $\{$alpha-digit$\}_0^7$ \\
  alpha-digit $::=$ \\
  \> alpha \\
  \> digit
\end{tabbing}
\end{quote}

\sect{2.2}{Comments and Character Sets}

Comments are delimited as in PL/I by \verb|/*| and \verb|*/|.

In general, B requires tokens to be separated by blanks, comments or
newlines, however the compiler infers separators surrounding any of
the characters \verb|(){][] ~;?:| or surrounding any maximal sequence
of the characters \verb$+-*/<>&|!$.

The character set used in B is ANSCII.

The syntactic variable `alpha' is not defined.  It represents the
characters \verb|A| to \verb|Z|, \verb|a| to \verb|z|, \verb|_|, and
backspace.

The syntactic variable `digit' is not defined.  It represents the
characters \verb|0|, \verb|1|, \ldots \verb|9|.

The syntactic variable `char' is not defined.  It is essentially any
character in the set plus the escape character `\verb|*|' followed by
another character to represent characters not easily represented in
the set.  The following escape sequences are currently defined:
\begin{quote}
\begin{tabular}{ll}
  \verb|*0| & null \\
  \verb|*e| & end-of-file \\
  \verb|*(| & \verb|{| \\
  \verb|*)| & \verb|}| \\
  \verb|*t| & tab \\
  \verb|**| & \verb|*| \\
  \verb|*'| & \verb|'| \\
  \verb|*"| & \verb|"| \\
  \verb|*n| & newline
\end{tabular}
\end{quote}

All keywords in the language are only recognized in lower case.
Keywords are reserved.

\sect{3.0}{Rvalues and Lvalues}

An rvalue is a binary bit pattern of a fixed length.  On the PDP-11 it
is 16 bits.  Objects are rvalues of different kinds such as integers,
labels, vectors and functions. The actual kind of object represented
is called the type of the rvalue.

A B expression can be evaluated to yield an rvalue, but its type is
undefined until the rvalue is used in some context.  It is then assumed
to represent an object of the required type.  For example, in the
following function call
\begin{quote}
  \verb|(b?f:g[i])(1,x>1)|
\end{quote}
The expression \verb|(b?f:g[i])| is evaluated to yield an rvalue which
is interpreted to be of type function.  Whether \verb|f| and
\verb|g[i]| are in fact functions is not checked.  Similarly, \verb|b| is
assumed to be of type truth value, \verb|x| to be type integer etc.

There is no check to insure that there are no type mismatches.
Similarly, there are no wanted, or unwanted, type conversions.

An lvalue is a bit pattern representing a storage location containing
an rvalue.  An lvalue is a type in B. The unary operator \verb|*| can
be used to interpret an rvalue as an lvalue. Thus
\begin{quote}
  \verb|*x|
\end{quote}
evaluates the expression \verb|x| to yield an rvalue, which is then
interpreted as an lvalue.  If it is then used in an rvalue context,
the application of \verb|*| yields the rvalue therein stored.  The
operator \verb|*| can be thought of as indirection.

The unary operator \verb|&| can be used to interpret an lvalue as an
rvalue. Thus
\begin{quote}
  \verb|&x|
\end{quote}
evaluates the expression \verb|x| as an lvalue.

The application of \verb|&| then yields the lvalue as an rvalue.  The
operator \verb|&| can therefore be thought of as the address function.
The names lvalue and rvalue come from the assignment statement which
requires an lvalue on the left and an rvalue on the right.

\sect{4.0}{Expression Evaluation}

Binding of expressions (lvalues and rvalues) is in the same order as
the sub-sections of this section except as noted.  Thus expressions
referred to as operands of `\verb|+|' (section 4.4) are expressions
defined in sections 4.1 to 4.3. The binding of operators at the same
level (left to right, right to left) is specified in each sub-section.

\sect{4.1}{Primary Expressions}
\begin{enumerate}
\item A name is an lvalue of one of three storage classes (automatic,
  external and internal).
\item A decimal constant is an rvalue.  It consists ot a digit
  between 1 and 9 followed by any number of digits between O and 9.
  The value of the constant should not exceed the maximum value that
  can be stored in an object.
\item An octal constant is the same as a decimal constant except that
  it begins with a zero.  It is then interpreted in base 8. Note that
  09 (base 8) is legal and equal to 011.
\item A character constant is represented by \verb|'| followed by one
  or two characters (possibly escaped) followed by another \verb|'|.

  It has an rvalue equal to the value of the characters packed and
  right adjusted.
\item A string is any number of characters between \verb|"|
  characters.  The characters are packed into adjacent objects
  (lvalues sequential) and terminated with the character ‘\verb|*e|’.
  The rvalue of the string is the lvalue of the object containing the
  first character.  See section 8.0 for library functions used to
  manipulate strings in a machine independent fashion.
\item Any expression in \verb|()| parentheses is a primary expression.
  Parentheses are used to alter order of binding.
\item A vector is a primary expression followed by any expression in
  \verb|[]| brackets.  The two expressions are evaluated to rvalues,
  added and the result is used as an lvalue.  The primary expression
  can be thought of as a pointer to the base of a vector, while the
  bracketed expression can be thought of as the offset in the vector.
  Since \verb|E1[E2]| is identical to \verb|*(E1+E2)|, and addition is
  commutative, the base of the vector and the offset in the vector can
  swap positions.
\item A function is a primary expression followed by any number of
  expressions in \verb|()| parentheses separated by commas.  The
  expressions in parentheses are evaluated (in an unspecified order)
  to rvalues and assigned to the function's parameters.  The primary
  expression is evaluated to an rvalue (assumed to be type function).
  The function is then called. Each call is recursive at little cost
  in time or space.
\end{enumerate}

Primary expressions are bound left to right.

\sect{4.2}{Unary Operators}

\begin{enumerate}
\item The rvalue (or indirection) prefix unary operator \verb|*| is
  described in section 3.0.  Its operand is evaluated to rvalue, and
  then used as an lvalue.  In this manner, address arithmetic may be
  performed.
\item The lvalue (or address) prefix unary operator \verb|&| is also
  described in section 3.0.  Note that \verb|&*x| is identically
  \verb|x|, but \verb|*&x| is only \verb|x| if \verb|x| is an lvalue.
\item The operand of the negate prefix unary operator \verb|-| is
  interpreted as an rvalue. The result is an rvalue with
  opposite sign.
\item The NOT prefix unary operator \verb|!| takes an integer rvalue
  operand.  The result is zero if the operand is non-zero.  The result
  is one if the operand is zero.
\item The increment \verb|++| and decrement \verb|--| unary operators
  may be used either in prefix or postfix form.  Either form requires
  an lvalue operand.  The rvalue stored in the lvalue is either
  incremented or decremented by one.  The result is the rvalue either
  before or after the operation depending on postfix or prefix
  notation respectively.  Thus if \verb|x| currently contains the
  rvalue 5, then \verb|++x| and \verb|x++| both change \verb|x| to 6.
  The value of \verb|++x| is 6 while \verb|x++| is 5. Similarly,
  \verb|--x| and \verb|x--| store 4 in \verb|x|. The former has rvalue
  result 4, the latter 5.
\end{enumerate}

Unary operators are bound right to left.  Thus \verb|-!x++| is bound
\verb|-(!(x++))|.

\sect{4.3}{Multiplicative Operators}

The multiplicative binary operators \verb|*|, \verb|/|, and \verb|%|,
expect rvalue integer operands.  The result is also an integer.

\begin{enumerate}
\item The operator \verb|*| denotes multiplication.
\item The operator / denotes division.  The result is correct if the
  first operand is divisible by the second.  If both operands are
  positive, the result is truncated toward zero.  Otherwise the
  rounding is undefined, but never greater than one.
\item The operator \verb|%| denotes modulo.  If both operands are
  positive, the result is correct. It is
  undefined otherwise.
\end{enumerate}

The multiplicative operators bind left to right.

\sect{4.4}{Additive Operators}

The binary operators \verb|+| and \verb|-| are add and subtract.  The
additive operators bind left to right.

\sect{4.5}{Shift Operators}

The binary operators \verb|<<| and \verb|>>| are left and right shift
respectively.  The left rvalue operand is taken as a bit pattern.  The
right operand is taken as an integer shift count.  The result is the
bit pattern shifted by the shift count. Vacated bits are filled with
zeros. The result is undefined if the shift count is negative or
larger than an object length. This shift operators bind left to right.

\sect{4.6}{Relational Operators}

The relational operators \verb|<| (less than), \verb|<=| (less than or
equal to), \verb|>| (greater than), and \verb|>=| (greater than or
equal to) take integer rvalue operands.  The result is one if the
operands are in the given relation to one another.  The result is zero
otherwise.

\sect{4.7}{Equality Operators}

The equality operators \verb|==| (equal to) and \verb|!=| (not equal
to) perform similarly to the relation operators.

\sect{4.8}{AND operator}

The AND operator \verb|&| takes operands as bit patterns.  The result
is the bit pattern that is the bit-wise AND of the operands.  The AND
operator binds and evaluates left to right.

\sect{4.9}{OR operator}

The OR operator \verb$|$ performs exactly as AND, but the result is
the bit-wise inclusive OR of the operands.  The OR operator also binds
and evaluates left to right.

\sect{4.10}{Conditional Expression}

Three rvalue expressions separated by \verb|?| and \verb|:| form a
conditional expression.  The first expression (to the left of the
\verb|?|) is evaluated.  If the result is non-zero, the second
expression is evaluated and the third ignored.  If the value is zero,
the second expression is ignored and the third is evaluated.  The
result is either the evaluation of the second or third expression.

Binding is right to left. Thus \verb|a?b:c?d:e| is \verb|a?b:(c?d:e)|.

\sect{4.11}{Assignment Operators}

There are 16 assignment operators in B. All have the form
\begin{quote}
  lvalue op rvalue
\end{quote}
The assignment operator \verb|=| merely evaluates the rvalue and
stores the result in the lvalue.  The assignment operators \verb$=|$,
\verb|=&|, \verb|===|, \verb|=!=|, \verb|=<|, \verb|=<=|, \verb|=>|,
\verb|=>=|, \verb|=<<|, \verb|=>>|, \verb|=+|, \verb|=-|, \verb|=%|,
\verb|=*| and \verb|=/| perform a binary operation (see sections 4.3
to 4.9) between the rvalue stored in the assignment's lvalue and the
assignment's rvalue. The result is then stored in the lvalue. The
expression \verb|x=*10| is identical to \verb|x=x*10|. Note that this
is not \verb|x= *10|. The result of an assignment is the
rvalue. Assignments bind right to left, thus \verb|x=y=0| assigns zero
to \verb|y|, then \verb|x|, and returns the rvalue zero.

\sect{5.0}{Statements}

Statements define program execution. Each statement is executed by the
computer in sequence. There are, of course, statements to
conditionally or unconditionally alter normal sequencing.

\sect{5.1}{Compound Statement}

A sequence of statements in \verb|{}| braces is syntactically a single
statement. This mechanism is provided so that where a single statement
is expected, any number of statements can be placed.

\sect{5.2}{Conditional Statement}

A conditional statement has two forms. The first:
\begin{quote}
  \token{if}\verb|(|rvalue\verb|)| statement$_1$
\end{quote}
evaluates the rvalue and executes statement$_1$, if the rvalue is
non-zero. If the rvalue is zero, statement$_!$ is skipped. The second
form:
\begin{quote}
  \token{if}\verb|(|rvalue\verb|)| statement$_1$ \token{else} statement$_2$
\end{quote}
is defined as follows in terms of the first form:
\begin{quote}
  \token{if}\verb|(x=(|rvalue\verb|))| statement$_1$ \token{if}\verb|(!x)| statement$_2$
\end{quote}
Thus, only one of the two statements is executed, depending on the
value of rvalue. In the above example, \verb|x| is not a real variable, but
just a demonstration aid.

\sect{5.3}{While Statement}

The while statement has the form:
\begin{quote}
  \token{while}\verb|(|rvalue\verb|)| statement
\end{quote}
The execution is described in terms of the conditional and goto
statements as follows:
\begin{quote}
  \verb|x:| \token{if}\verb|(|rvalue\verb|)| \verb|{| statement \token{goto} \verb|x;| \verb|}|
\end{quote}
Thus the statement is executed repeatedly while the rvalue is
non-zero. Again, \verb|x| is a demonstration aid.

\sect{5.4}{Switch Statement}

The switch statement is the most complicated statement in B. The
switch has the form:
\begin{quote}
  \token{switch} rvalue statement$_1$
\end{quote}
Virtually always, statement$_1$ above is a compound statement. Each
statement in statement$_1$ may be preceded by a case as follows:
\begin{quote}
  \token{case} constant \token{:}
\end{quote}
During execution, the rvalue is evaluated and compared to each case
constant in undefined order. If a case constant is equal to the
evaluated rvalue, control is passed to the statement following the
case. If the rvalue matches none of the cases, statement$_1$ is skipped.

\sect{5.5}{Goto Statement}

The goto statement is as follows:
\begin{quote}
  \token{goto} rvalue \token{;}
\end{quote}

The rvalue is expected to be of type label. Control is then passed to
the corresponding label. Goto's cannot be executed to labels outside
the currently executing function level.

\sect{5.6}{Return Statement}

The return statement is used in a function to return control to the
caller of a function. The first form simply returns control.
\begin{quote}
  \token{return} \token{;}
\end{quote}

The second form returns an rvalue for the execution of the function.
\begin{quote}
  \token{return} \token{(} rvalue \token{)} \token{;}
\end{quote}

The caller of the function need not use the returned rvalue.

\sect{5.7}{Rvalue Statement}

Any rvalue followed by a semicolon is a statement. The two most common
rvalue statements are assignment and function call.

\sect{5.8}{Null Statement}

A semicolon is a null statement causing no execution. It is used
mainly to carry a label after the last executable statement in a
compound statement. It sometimes has use to supply a null body to a
while statement.

\sect{6.0}{Declarations}

Declarations in B specify storage class of variables. Such
declarations also, in some circumstances, specify initialization.

There are three storage classes in B. Automatic storage is allocated
for each function invocation. External storage is allocated before
execution and is available to any and all functions. Internal storage
is local to a function and is available only to that function, but is
available to all invocations of that function.

\sect{6.1}{External Declaration}

The external declaration has the form:
\begin{quote}
  \token{extrn} name$_1$ \token{,} name$_2$ \ldots \token{;}
\end{quote}

The external declaration specifies that each of the named variables is
of the external storage class. The declaration must occur before the
first use of each of the variables. Each of the variables must also be
externally defined.

\sect{6.2}{Automatic Declaration}

The automatic declaration also constitutes a definition:
\begin{quote}
  \token{auto} name$_1^{}$ $\{$constant$\}_0^1$ \token{,}
name$_2^{}$ $\{$constant$\}_0^1$ \ldots \token{;}
\end{quote}
  
In absence of the constant, the automatic declaration defines the
variable to be of class automatic. At the same time, storage is
allocated for the variable. When an automatic declaration is followed
by a constant, the automatic variable is also initialized to the base
of an automatic vector of the size of the constant. The actual
subscripts used to reference the vector range from zero to the value
of the constant less one.
          
\sect{6.3}{Internal Declaration}

The first reference to a variable not declared as external or
automatic constitutes an internal declaration, All internal variables
not defined as labels are flagged as undefined within a
function. Labels are defined and initialized as follows:
\begin{quote}
  name \token{:}
\end{quote}

\sect{7.0}{External Definitions}

A complete B program consists of a series of external
definitions. Execution is started by the hidden sequence
\begin{quote}
  \verb|main(); exit();|
\end{quote}

Thus, it is expected that one of the external definitions is a
function definition of main. (Exit is a predefined library
function. See section 8.0)
        
\sect{7.1}{Simple Definition}

The simple external definition allocates an external object and
optionally initializes it:
\begin{quote}
  name $\{$ival \token{,} ival \ldots$\}_0^{}$ \token{;}
\end{quote}
If the external object is defined with no initialization, it is
initialized with zero. A single initialization with a constant
initializes the external with the value of the
constant. Initialization with a name initializes the external to the
address of that name. Many such initializations may be accessed as a
vector based at \verb|&|name.
        
\sect{7.2}{Vector Definitions}

An external vector definition has the following form:
\begin{quote}
  name \token{[} $\{$constant$\}_0^1$ \token{]}
  $\{$ival \token{,} ival \ldots$\}_0^{}$ \token{;}
\end{quote}
The name is initialized with the lvalue of the base of an external
vector. If the vector size is missing, zero is assumed. In either
case, the vector is initialized with the list of initial values. Each
initial value is either a constant or a name. A constant initial value
initializes the vector element to the value of the constant. The name
initializes the element to the address of the name. The actual size is
the maximum of the given size and the number of initial values.

\sect{7.3}{Function Definitions}

Function definitions have the following form:
\begin{quote}
  name \token( arguments \token) statement
\end{quote}

The name is initialized to the rvalue of the function. The arguments
consist of a list of names separated by commas. Each name defines an
automatic lvalue that is assigned the rvalue of the corresponding
function call actual parameters. The statement (often compound)
defines the execution of the function when invoked.
        
\sect{8.0}{Library Functions}

There is a library of B functions maintained in the file
\verb|/etc/libb.a|. The following is a list of those functions
currently in the library. See section II of [4] for complete
descriptions of the functions marked with an *.
\begin{description}
\item[]\f{c} = char(\f{string}, \f{i});\hfill\\
  The \f{i}-th character of the \f{string} is returned.
\item[]\f{error} = chdir(\f{string});\hfill\\
  The path name represented by the \f{string} becomes the current
  directory. A negative number returned indicates an error. (*)
\item[]\f{error} = chmod(\f{string}, \f{mode});\hfill\\
  The file specified by the \f{string} has its mode changed to the \f{mode}
  argument. A negative number returned indicates an error. (*)
\item[]\f{error} = chown(\f{string}, \f{owner}); \hfill\\
  The file specified by the \f{string} has its owner changed to the \f{owner}
  argument. A negative number returned indicates an error. (*)
\item[]\f{error} = close(\f{file}); \hfill\\
  The open file specified by the \f{file} argument is closed. A negative
  number returned indicates an error. (*)
\item[]\f{file} = creat(\f{string}, \f{mode}); \hfill\\
  The file specified by the \f{string} is either truncated or created in
  the \f{mode} specified depending on its prior existence. In both cases,
  the file is opened for writing and a file descriptor is returned. A
  negative number returned indicates an error. (*)
\item[]ctime(\f{time}, \f{date}); \hfill\\
  The system time (60-ths of a second) represented in the two-word
  vector \f{time} is converted to a 16-character date in the 8-word
  vector \f{date}. The converted date has the following format:
  ``\verb|Mmm dd hh:mm:ss|''.
\item[]execl(\f{string}, $\f{arg}_0$, $\f{arg}_1$, \ldots, 0); \hfill\\
  The current process is replaced by the execution of the file
  specified by \f{string}. The $\f{arg}_i$ strings are passed as arguments. A
  return indicates an error. (*)
\item[]execv(\f{string}, \f{argv}, \f{count}); \hfill\\
  The current process is replaced by the execution of the file
  specified by \f{string}. The vector of strings of length \f{count} are
  passed as arguments. A return indicates an error. (*)
\item[]exit(); \hfill\\
  The current process is terminated. (*)
\item[]error = fork(); \hfill\\
  The current process splits into two. The child process is returned a
  zero. The parent process is returned the process ID of the child. A
  negative number returned indicates an error. (*)
\item[]\f{error} = fstat(\f{file}, \f{status}); \hfill\\
  The i-node of the open file designated by \f{file} is put in the 20-word
  vector \f{status}. A negative number returned indicates an error. (*)
\item[]\f{char} = getchar();\hfill\\
  The next character from the standard input file is returned. The
  character `\verb|*e|' is returned for an end-of-file.
\item[]\f{id} = getuid();\hfill\\
  The user-ID of the current process is returned. (*)
\item[]\f{error} = gtty(\f{file}, \f{ttystat});\hfill\\
  The teletype modes of the open file designated by \f{file} is returned
  in the 3-word vector \f{ttystat}. A negative number returned indicates an
  error. (*)
\item[]lchar(\f{string}, \f{i}, \f{char}); \hfill\\
  The character \f{char} is stored in the \f{i}-th character of the \f{string}.
\item[]\f{error} = link($\f{string}_1$, $\f{string}_2$); \hfill\\
  The pathname specified by $\f{string}_2$ is created such that it is a link
  to the existing file specified by $\f{string}_1$. A negative number
  returned indicates an error. (*)
\item[]\f{error} = mkdir(\f{string}, \f{mode}); \hfill\\
  The directory specified by the \f{string} is made to exist with the
  specified access \f{mode}. A negative number returned indicates an
  error. (*)
\item[]\f{file} = open(\f{string}, \f{mode}); \hfill\\
  The file specified by the \f{string} is opened for reading if \f{mode} is
  zero, for writing if mode is not zero. The open file designator is
  returned. A negative number returned indicates an error. (*)
\item[]printf(\f{format}, $\f{arg}_1$, \ldots); \hfill\\
  See section 9.3 below.
\item[]printn(\f{number}, \f{base}); \hfill\\
  See section 9.1 below.
\item[]putchar(\f{char}); \hfill\\
  The character \f{char} is written on the standard output file.
\item[]\f{nread} = read(\f{file}, \f{buffer}, \f{count}); \hfill\\
  \f{Count} bytes are read into the vector \f{buffer} from the open file
  designated by \f{file}. The actual number of bytes read are returned. A
  negative number returned indicates an error. (*)
\item[]\f{error} = seek(\f{file}, \f{offset}, \f{pointer}); \hfill\\
  The I/O pointer on the open file designated by \f{file} is set to the
  value of the designated \f{pointer} plus the \f{offset}. A pointer of zero
  designates the beginning of the file. A pointer of one designates
  the current I/O pointer. A pointer of two designates the end of the
  file. A negative number returned indicates an error. (*)
\item[]\f{error} = setuid(\f{id}); \hfill\\
  The user-ID of the current process is set to \f{id}. A negative number
  returned indicates an error. (*)
\item[]\f{error} = stat(\f{string}, \f{status}); \hfill\\
  The i-node of the file specified by the \f{string} is put in the 20-word
  vector \f{status}. A negative number returned indicates an error. (*)
\item[]\f{error} = stty(\f{file}, \f{ttystat}); \hfill\\
  The teletype modes of the open file designated by \f{file} is set from
  the 3-word vector \f{ttystat}. A negative number returned indicates an
  error. (*)
\item[]time(\f{timev}); \hfill\\
  The current system time is returned in the 2-word vector \f{timev}. (*)
\item[]\f{error} = unlink(\f{string}); \hfill\\
  The link specified by the \f{string} is removed. A negative number
  returned indicates an error. (*)
\item[]\f{error} = wait(); \hfill\\
  The current process is suspended until one of its child processes
  terminates. At that time, the child's process-ID is returned. A
  negative number returned indicates an error. (*)
\item[]\f{nwrite} = write(\f{file}, \f{buffer}, \f{count}); \hfill\\
  \f{Count} bytes are written out of the vector \f{buffer} on the open file
  designated by \f{file}. The actual number of bytes written are
  returned. A negative number returned indicates an error. (*)
\end{description}

Besides the functions available from the library, there is a
predefined external vector named \verb|argv| included with every
program. The size of argv is \verb|argv[0]|+1. The elements
\verb|argv[1]|\ldots \verb|argv[argv[0]]| are the parameter strings
as passed by the system in the execution of the current process. (See
shell in II of [4])

\sect{9.0}{Examples}

The examples appear exactly as given to B.

\sect{9.1}{}

\begin{verbatim}
/* The following function will print a non-negative number, n, to
   the base b, where 2<=b<=10,  This routine uses the fact that
   in the ANSCII character set, the digits O to 9 have sequential
   code values.  */

   printn(n,b) {
      extrn putchar;
      auto a;

      if(a=n/b) /* assignment, not test for equality */
         printn(a, b); /* recursive */
      putchar(n%b + '0');
   }
\end{verbatim}

\sect{9.2}{}
\begin{verbatim}
/* The following program will calculate the constant e-2 to about
   4000 decimal digits, and print it 50 characters to the line in
   groups of 5 characters.  The method is simple output conversion
   of the expansion
     1/2! + 1/3! + ... = .111....
   where the bases of the digits are 2, 3, 4, . . . */

   main() {
      extrn putchar, n, v;
      auto i, c, col, a;

      i = col = 0;
      while(i<n)
         v[i++] = 1;
      while(col<2*n) {
         a = n+1 ;
         c = i = 0;
         while (i<n) {
            c =+ v[i] *10;
            v[i++]  = c%a;
            c =/ a--;
         }
         putchar(c+'0');
         if(!(++col%5))
            putchar(col%50?' ': '*n');
      }
      putchar('*n*n');
   }

   v[2000];
   n 2000;
\end{verbatim}

\sect{9.3}{}

\begin{verbatim}
/* The following function is a general formatting, printing, and
   conversion subroutine.  The first argument is a format string.
   Character sequences of the form `%x' are interpreted and cause
   conversion of type 'x' of the next argument, other character
   sequences are printed verbatim.  Thus

      printf("delta is %d*n", delta);

   will convert the variable delta to decimal (%d) and print the
   string with the converted form of delta in place of %d.  The
   conversions %d-decimal, %o-octal, *s-string and %c-character
   are allowed.

   This program calls upon the function 'printn'. (see section
   9.1) */

   printf(fmt, x1,x2,x3,x4,x5,x6,x7,x8,x9) {
      extrn printn, char, putchar;
      auto adx, x, c, i, j;

      i= 0; /* fmt index */
      adx = &x1; /* argument pointer */
   loop:
      while((c=char(fmt,i++) ) != `%') {
         if(c == `*e')
            return;
         putchar(c);
      }
      x = *adx++;
      switch c = char(fmt,i++) {

      case `d': /* decimal */
      case `o': /* octal */
         if(x < O) {
            x = -x;
            putchar('-');
         }
         printn(x, c=='o'?8:1O);
         goto loop;

      case 'c': /* char */
         putchar(x);
         goto loop;

      case 's': /* string */
         j = 0;
         while((c=char(x,j++)) != '*e')
            putchar(c);
         goto loop;
      }
      putchar('%');
      i--;
      adx--;
      goto loop;
   }
\end{verbatim}

\sect{10.0}{Usage}

Currently on UNIX, there is no B command. The B compiler phases must
be executed piecemeal. The first phase turns a B source program into
an intermediate language.
\begin{quote}
  \verb|/etc/bc source interm|
\end{quote}
The next phase turns the intermediate language into assembler source,
at which time the intermediate language can be removed.
\begin{quote}
  \verb|/etc/ba interm asource| \\
  \verb|rm interm|
\end{quote}
The next phase assembles the assembler source into the object file
\verb|a.out|. After this the \verb|a.out| file can be renamed and the
assembler source file can be removed.
\begin{quote}
  \verb|as asource| \\
  \verb|mv a.out object| \\
  \verb|rm asource|
\end{quote}
The last phase loads the various object files with the necessary
libraries in the desired order.
\begin{quote}
  \verb|ld object /etc/brt1 -lb /etc/bilib /etc/brt2|
\end{quote}
Now \verb|a.out| contains the completely bound and loaded program and
can be executed.
\begin{quote}
  \verb|a.out|
\end{quote}
A canned sequence of shell commands exists invoked as follows:
\begin{quote}
  \verb|sh /usr/b/rc x|
\end{quote}
It will compile, convert, assemble and load the file \verb|x.b| into the
executable file \verb|a.out|.

\sect{12.0}{Implementation and Debugging\footnote{There is no section 11.0. [ed]}}

A B program is implemented as a reverse Polish threaded code
interpreter: The object code consists of a series of addresses of
interpreter subroutines. Machine register 3 is dedicated as the
interpreter program counter. Machine register 4 is dedicated as the
interpreter display pointer. The display pointer points to the base of
the current stack frame. The first word of each stack frame is a
pointer to the previous stack frame (prior display pointer.) The
second word in each frame is is the saved interpreter program counter
(return point of the call creating the frame.) Automatic variables
start at the third word of each frame. Machine register 5 is dedicated
as the interpreter stack pointer. The machine stack pointer plays no
role in the interpretation. An example source code segment, object
code and interpreter subroutines follow:
\begin{verbatim}
      automatic = external + 100.;

      va; 4                   / lvalue of first automatic on stack
      x; .external            / rvalue of external on stack
      c; 100.                 / rvalue of constant on stack
      b12                     / binary operator #12 (+)
      b1                      / binary operator #1 (=)
      ...
va:
      mov     (r3)+,r0
      add     r4,r0           / dp+offset of automatic
      asr     r0              / lvalues are word addresses
      mov     r0,(r5)+
      jmp     *(r3)+          / linkage between subroutines
x:
      mov     *(r3)+,(r5)+
      jmp     *(r3)+
c:
      mov     (r3)+,(r5)+
      jmp     *(r3)+

b12:
      add     -(r5),-2(r5)
      jmp     *(r3)+

b1:
      mov     -(r5),r0        / rvalue
      mov     -(r5),r1        / lvalue
      asl     r1              / now byte address
      mov     r0,(r1)         / actual assignment
      mov     r0,(r5)+        / = returns an lvalue
      jmp     *(r3)+
\end{verbatim}

The above code as compared to the obvious 3 instruction directly
executed equivalent gives the approximate 5:1 speed and 2:1 space
penalties one pays in using B.

The salient features for debugging are then:
\begin{enumerate}
\item Machine r4 is the display pointer and can be used to trace
  function calls and determine automatic variable values at each call.
\item Machine r3 is the current program counter and can be used to
  determine the current point of execution.
\item All externals are globals with their variable names prefixed by
  `.'. Thus the debugger [4] can be used directly to give values of
  external variables.
\item All data lvalues are word addresses and therefore not directly
  examinable by the debugger. (See ` request in I of [4])
\end{enumerate}
  
\sect{13.0}{Nasties}

This section describes the `glitches' found in all languages, but
rarely reported.
\begin{enumerate}
\item The compiler makes sense of certain expressions with operators
  in ambiguous cases (e.g. \verb|a+++b|) but not others even in unambiguous
  cases (e.g. \verb|a+++++b|).
\item The B assembler \verb|/etc/ba| does not correctly handle all possible
  combinations of intermediate language. The symptom is undefined
  symbols in the assembly of the output from \verb|/etc/ba|. This is rare.
\item The B interpreter \verb|/etc/bilib| is really a library of threaded
  code segment. The following code segments have not yet been written:
  \begin{quote}
    \begin{tabular}{ll}
      b103 & \verb|=\&| \\
      b104 & \verb|===| \\
      b105 & \verb|=!=| \\
      b106 & \verb|=<=| \\
      b107 & \verb|=<| \\
      b110 & \verb|=>=| \\
      b111 & \verb|=>| \\
      b120 & \verb|=/| \\
    \end{tabular}
  \end{quote}

\item Initialization of external variables with addresses of other
  externals is not possible due to a loader deficiency.
\item Since externals are implemented as globals with names preceded
  by `\verb|.|', the external names `\verb|byte|', `\verb|endif|',
  `\verb|even|' and `\verb|globl|' conflict with assembler
  pseudooperations and should be avoided.
\end{enumerate}

\sect{14.0}{Diagnostics}

Diagnostics consist of two letters, an optional name, and a source
line number. Due to the free format of the source, the number might be
high. The following is a list of the diagnostics.
\begin{quote}
  \begin{tabular}{lll}
    \toprule
    error & name & meaning \\
    \midrule
    \verb|$)| & --- & \verb|{}| imbalance \\
    \verb|()| & --- & \verb|()| imbalance \\
    \verb|*/| & --- & \verb|/* */| imbalance \\
    \verb|[]| &	--- & \verb|[]| imbalance \\
    \verb|>c| & --- & case table overflow (fatal) \\
    \verb|>e| & --- & expression stack overflow (fatal) \\
    \verb|>i| & --- & label table overflow (fatal) \\
    \verb|>s| & --- & symbol table overflow (fatal) \\
    \verb|ex| & --- & expression syntax \\
    \verb|lv| & --- & rvalue where lvalue expected \\
    \verb|rd| &	name & name redeclaration \\
    \verb|sx| & keyword & statement syntax \\
    \verb|un| & name & undefined name \\
    \verb|xx| &	--- & external syntax \\
    \bottomrule
  \end{tabular}
\end{quote}

\vspace{1cm}

\begin{tabular}{p{9cm}p{4cm}}
  MH-1271-KT-pdp \newline
  Attached References
  & \raggedright K. Thompson \newline
    \vspace{0cm}
    \includegraphics[scale=0.25]{sig.png}  
\end{tabular}

\sect{}{References}

\begin{enumerate}
\item[1.] Richards, M. \emph{The BCPL Reference Manual}. Multics
  repository M0099.
\item[2.] Canaday, R.H. and Ritchie, D.M. \emph{Bell Laboratories BCPL}. MM
  69-1371/1373-7/12.
\item[3.] Ritchie, D.M. \emph{The UNIX Time Sharing System}. MM 71-1273-4.
\item[4.] Thompson, K. and Ritchie, D.M. \emph{UNIX Programmer's
    Manual}. Available by arrangement.
\end{enumerate}

\end{document}

%%% Local Variables:
%%% mode: latex
%%% TeX-master: t
%%% End:
